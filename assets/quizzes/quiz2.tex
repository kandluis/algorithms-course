% !TEX options=--shell-escape
\documentclass [12pt]{article} 
\usepackage {amsmath}
\usepackage {amsthm}
\usepackage {amssymb}
\usepackage {graphicx} 
\usepackage {float}
\usepackage {multirow}
\usepackage {xcolor}
\usepackage {algorithmic}
\usepackage [ruled,vlined,commentsnumbered,titlenotnumbered]{algorithm2e} \usepackage {array} 
\usepackage {booktabs} 
\usepackage {url} 
\usepackage {parskip} 
\usepackage [margin=1in]{geometry} 
\usepackage [T1]{fontenc} 
\usepackage {cmbright} 
\usepackage [many]{tcolorbox} 
\usepackage [colorlinks = true,
            linkcolor = blue,
            urlcolor  = blue,
            citecolor = blue,
            anchorcolor = blue]{hyperref} 
\usepackage {enumitem} 
\usepackage {xparse} 
\usepackage {verbatim}
\usepackage{listings}
\usepackage{xcolor}
\usepackage{csquotes}
\usepackage[cache=false]{minted}
\usepackage{mdframed}
\usepackage{tikz}
\usetikzlibrary{shapes.symbols}
\newtheorem{theorem}{Theorem}

\DeclareTColorBox {Solution}{}{breakable, title={Solution}}
\DeclareTColorBox {Solution*}{}{breakable, title={Solution (provided)}}
\DeclareTColorBox {Instruction}{}{boxrule=0pt, boxsep=0pt, left=0.5em, right=0.5em, top=0.5em, bottom=0.5em, arc=0pt, toprule=1pt, bottomrule=1pt}
\DeclareDocumentCommand {\Expecting }{+m}{\textbf {[We are expecting:} #1\textbf {]}}
\DeclareDocumentCommand {\Points }{m}{\textbf {(#1 pt.)}} 
\newcommand {\hint }[1]{\noindent {[\textbf {HINT:} \em #1 \em ]}} \newcommand {\pts }[1]{\textbf {(#1 pt.)}} 

\begin{document} 

{\LARGE \textbf {COMP 285 (NC A\&T, Spr `22)}\hfill \textbf {Weekly Quiz 2} } 

\section{} Which of the following is the correct recurrence relations for MergeSort?

\begin{Solution}
$T(n) = 2 \cdot T\left(\frac{n}{2}\right) + O(n)$
\paragraph{} 
 For each step, MergeSort divides the original problem by two, recursively calling itself to solve these two smaller problems. This is where $2 \cdot T(n)$ comes from. Once it has the answers, it merges the results which takes an additional $O(n)$ time, giving the recurrence above.
\end{Solution}


\section{} What's the closed-form solution for the running time of the following recurrence relation $T(n) = 5 \cdot T\left(\frac{n}{3}\right) + O(n)$ (Hint: You might want to use the Master Theorem) (Aside: This is the actual recurrence relation of Strassen's Multiplication Algorithm, an improvement to Karatsuba's)

\begin{Solution}
$$
O(n^{\log_3 5})
$$

We can see that $a=5, b=3, d=1, a=5 > b^d=3$, so the result would be  $O(n^{\log_b a})$ = $O(n^{\log_3 5})$ according to the Master Theorem.
\end{Solution}


\section{} What's the closed-form solution for the running time of the following recurrence relation $T(n) = T\left(\frac{999n}{1000}\right) + O(n)$ (Hint: You might want to use the Master Theorem).

\begin{Solution}
$$O(n)$$

$a=1, b=\frac{1000}{999}, d=1, a=1 < b^d=\frac{1000}{999}$, so the result would be $O(n)$ according to the Master Theorem.
\end{Solution}


\section{} Select the recurrence relations below for which you CANNOT directly apply the Master Theorem.

\begin{Solution}
\begin{itemize}
  \item $T(n) = 2T(n-1) + O(n^2)$ because were are creating smaller problems of size $n - 1$. The Master Theore only works when the problems become a fraction of their original size.
  \item $T(n) = 4T\left(\frac{9n}{10}\right) + O(n \log n)$ because the additional work we do to combine the problems is not polynomial (eg, $n^d$) but $n \log n$.
  \item $T(n) = T\left(\frac{n}{5}\right) + T\left(\frac{7n}{10}\right) + O(n)$ because we don't split the original problem into subproblems of equal size.
\end{itemize}
\end{Solution}


\section{} There is an $O(n)$ time algorithm for the k-Select Problem.

\begin{Solution}
Yes. Use divide-and-conquer recursive-based solution as we covered in class.
\end{Solution}


\section{} What is the running time of a mergesort-based solution to the k-Select problem?

\begin{Solution}
$$\Theta(nlogn)$$

MergeSort will take $\Theta(n \log n)$ time.
\end{Solution}


\section{} In our divide-and-conquer recursive-based solution to the k-Select problem, what is the running time if we always pick the minimum as the pivot.

\begin{Solution}
$$\Theta(n^2)$$
If we always pick the minimum(worst-case pivot), we are unable to divide the problem into half each time. The recurrence relation will be $T(n) = T(n-1) + O(n)$ which will end-up with a running time of $O(n^2)$.
\end{Solution}


\section{} In our divide-and-conquer recursive-based solution to the k-Select problem, what is the running time if we always pick the median as the pivot.

\begin{Solution}
$$\Theta(n)$$
\paragraph{} 
If we always pick the median(best-case pivot), we can divide the problem into half each time. The recurrence relation will be $T(n) = T\left(\frac{n}{2}\right) + O(n)$ which is $O(n)$ by the Master Theorem.
\end{Solution}


\section{} The running time of our trivial implementation of k-Select using MergeSort is always slower than the running time of a divide-and-conquer solution that randomly selects the pivot element.

\begin{Solution}
False
\paragraph{} 
The running time of k-Select using MergeSort is $\Theta(n \log n)$, and in worst-case the divide-and-conquer solution would take $\Theta(n^2)$ time, which is slower than $\Theta(n \log n)$.
\end{Solution}


\section{} In practice, it's often best to simply pick the pivot randomly rather than implement a more sophisticated, deterministic pivot selection method.

\begin{Solution}
True
\paragraph{} 
If there is a bad guy who gets to see our pivot choices, that’s just as bad as the worst-case pivot.

\end{Solution}


















\end{document} 